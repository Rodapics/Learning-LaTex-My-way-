\documentclass[12pt, twocolums, letterpaper]{article}
\usepackage[lmargin=0.5cm,rmargin=1cm,top=0.5cm,bottom=0.5cm]{geometry}
\usepackage[utf8]{inputenc}
\usepackage[T1]{fontenc}
\usepackage[spanish]{babel}
%\parident=0cm
%%
\usepackage{tikz}
%%
\usepackage{quantikz}

\begin{document}
\twocolumn[]
En este documento haremos un resumen de las figuras de simuladores que podemos generar en Quantikz

\section{Usos Básicos}
Generaremos las lineas por las cuales los circuitos se conectan, estas lineas se denominan \textbf{Cables Cuánticos}:\\ \\
\begin{quantikz}
&&&&&
\end{quantikz}\\ 
El código usado para realiar esto es:


\begin{verbatim}
	\begin{quantikz}
	&&&&&
	\end{quantikz}
\end{verbatim}
Si necesitamos crear dos lineas de cables cuánticos, lo podemos realizar con el mismo código: \\ \\
\begin{quantikz}
&&&& \\
&&&&
\end{quantikz}\\ \\
Código usado:
\begin{verbatim}
	\begin{quantikz}
	&&&&& \\
	&&&&
	\end{quantikz}
\end{verbatim}
De la misma manera, si queremos generar un cable cuántico con diferente diseño, lo podemos especificar, como se muestra a continuación:

\end{document}