\documentclass[10pt,onecolumn,letterpaper]{article}
%---------------------Paquetes-----------------------------
%Paquete para columnas, falta adiccionar esto y añadir el ejemplo al texto, incluir ejemplo de dos columnas.
\usepackage[spanish]{babel}
\usepackage[latin1,utf8]{inputenc}
\usepackage{amssymb,amsmath} %Lenguaje matemático %amsmath, el de las integrales
\usepackage{verbatim} %Escribir código
\usepackage{bm}
\usepackage{graphics,graphicx,xcolor}
%% Sets page size and margins
\usepackage[a4paper,top=2cm,bottom=2cm,left=2cm,right=2cm,marginparwidth=1.75cm]{geometry}
%Fecha en español. Nota: \renewcommand. También paquete para email.
\usepackage{multirow}
\usepackage{multicol}
%---------------------------------------------------------------------
\title{Respuestas ejercicios propuestos para LaTex}
\author{CARLOS ANDRES RODALLEGA MILLAN}
\date{\today}
%++++++++++++++++++++++CUERPO DEL DOCUMENTO+++++++++++++++++++++++++++
\begin{document}
\maketitle 
%.-.-.-.-.
\section{Primer punto} 
Utilice los entornos \verb+\array+ o \verb+\pmatrix+ para recrear la siguiente matrix:
\begin{equation}
\begin{pmatrix}\nonumber
	F[1,1] & \cdots & F[1,m]\\
	\vdots & \ddots & \vdots\\
	F[n,1] & \cdots & F[n,m]
\end{pmatrix}
\end{equation}
Tenga en cuenta que para los puntos puesto en anterior matrix los código dados fueron los siguientes:
\begin{verbatim}
\cdots   Horizontales (centados)
\ldots	 Horizontales (al piso)
\vdots	 Verticales
\ddots	 Diagonales
\end{verbatim}
\section{Segundo Punto} 
Recrear exactamente el extracto dado. \\
\begin{center}
\textbf{Ejercicios sobre coordenadas cilíndricas}
\end{center}
La ecuación del cardiole está dada por $r=k(1+cos\theta)$ por lo tanto:
\begin{eqnarray*}
\dot{r}=-k\ddot{\theta}sen(\theta) \\
\ddot{r}= -k(\dot{\theta ^{2}} cos\theta + \ddot{\theta}sen\theta)
\end{eqnarray*}
En  coordenadas polares la velocidad de la partícula está dada por la expresión:
\begin{equation*}
\vec{v}=\dot{r}\hat{r}+r\dot{\theta} \hat{\theta}
\end{equation*}
Como la rapidez es constante entonces
\begin{eqnarray*}
V_0^{2}=\dot{r}^{2}+r^{2}\dot{\theta}^{2}=2k^{2}\dot{\theta^{2}}(1+cos\theta) \\
\Longrightarrow \fbox{$\dot{\theta} = \frac{V_0}{\sqrt{2k^{2}(1+cos\theta)}}=\frac{V_0}{\sqrt{2kr}}$}
\end{eqnarray*}
Para calcular la aceleración radial $a_r=\ddot{r}-r\ddot{\theta^{2}}$ necesitamos calcular $\ddot{\theta}$
\begin{equation*}
\dot{\theta}=\frac{V_0}{\sqrt{2kr}}\Longrightarrow \ddot{\theta}=-\frac{V_0\dot{r}}{2r\sqrt{2kr}}= \frac{ksen\theta}{2r}\dot{\theta^{2}}=\frac{V_0^{2}sen\theta}{4r^{2}}
\end{equation*}
Luego
\begin{gather*}
a_r=\ddot{r}-r\dot{\theta^{2}}=-kcos\theta\dot{\theta^{2}}- ksen\theta\ddot{\theta}-r\dot{\theta^{2}}\\
\Longrightarrow a_r=-k\dot{\theta^{2}}[cos\theta +\frac{sin^{2}\theta}{2(1+cos\theta)}+(1+cos\theta)]\\
a_r=-k\dot{\theta^{2}}[\frac{2cos\theta+2cos^{2}\theta+ sen^{2}\theta +2+4cos\theta+2cos^{2}\theta}{2(1+cos\theta)}]\\
a_r=\frac{-k\dot{\theta}^{2}}{2(1+cos\theta)}(4cos^{2}\theta + sen^{2}\theta +6cos\theta + 2)=\frac{-3k\dot{\theta^{2}}}{2(1+cos\theta)}(1+cos\theta)^{2}\\ 
\\
\fbox{$a_r=-\frac{3}{4}\frac{V_0^{2}}{k}$}
\end{gather*}
\section{Tercer Punto}
Reproduzca el siguiente extracto de documento usando LaTex.
\\
\begin{center}
\textbf{Let $\theta,\beta$ be $3x3$ skew-symmetric matrices and $\sigma$ be a $3x3$ matrix. Find symmetric $S,T$ such that:}
\end{center}
$$(S-\theta)(T-\beta)=\sigma$$
After all previous consideratios, to find one solution we assume that T is diagonal.\newline
\\
Denote R= $\begin{pmatrix}\nonumber
	0 & c & b\\
	-c & 0 & a\\
	-b & -a & 0
\end{pmatrix}$, A=$\begin{pmatrix}\nonumber
	m_1 & m_2 & m_3\\
	m_4 & m_5 & m_6\\
	m_7 & m_8 & m_)
\end{pmatrix}$, Q=$\begin{pmatrix}\nonumber
	0 & q_1 & q_2\\
	-q_1 & 0 & q_3\\
	-q_2 & -q_3 & 0
\end{pmatrix}$, T=$\begin{pmatrix}\nonumber
	x & 0 & 0\\
	0 & y & 0\\
	0 & 0 & z
\end{pmatrix}$.\newline
\\
\\
\\
\\
Then we have TRT=$\begin{pmatrix}\nonumber
	0 & c & b\\
	-c & 0 & a\\
	-b & -a & 0
\end{pmatrix}$, $A^{\dagger}T=\begin{pmatrix}\nonumber
	0 & c & b\\
	-c & 0 & a\\
	-b & -a & 0
\end{pmatrix}$ 
\\
\\
Then we have the explicit form of the equation:
\begin{equation}\label{eq1}
cxy+m_4y-m_2x=q_1\\
\end{equation}

\begin{equation}\label{eq2}
bxz+m_7z-m_3x=q_2\\
\end{equation}

\begin{equation}\label{eq3}
ayz+m_8z-m_6y=q_3\\
\end{equation}
\\
This system of equations is solved by eliminate z (by (\ref{eq2}) and (\ref{eq3})) then calculate y from x (by the identity of
xy). Then we are left with a quadratic equation of x.\newline
Have x we can solve y, z.\newline
The explicit solution is obtainable but not worth calculated by hand.
\newpage
\section{Punto 4}
Here are the formulas for counting in various ways:
\begin{table}[h!]
\begin{center}
\begin{tabular}{|l||c|c|}
    \hline
     &No Repetiton&Repetition Allowed \\ \hline \hline
    \multirow{3}{*}{Not sensitive to order}& & \\
    &$\left(\begin{array}{c}
     n\\
     r
\end{array}\right)$&$\left(\left(\begin{array}{c}
     n\\
     r
\end{array}\right)\right)=\left(\begin{array}{c}
     n+r-1\\
     r
\end{array}\right)$\\
    & & \\\hline
        \multirow{3}{*}{Sensitive to order}& & \\
    &$P(n,r)$&$n^{r}$\\
    & & \\
    \hline
\end{tabular}\caption{Primera tabla de la tarea :D}
\end{center}
\label{tab:multicol}
\end{table}\\

Here are examples to demonstrate:
\\
\begin{table}[h!]
\begin{center}
\begin{tabular}{|l||c|c|}
    \hline
     &No Repetiton&Repetition Allowed \\ \hline \hline
    \multirow{5}{*}{Not sensitive to order}& & \\
    &5 distinct books &unlimited copies of 5 books\\
    &choose 3 books to take home &choose 3 copies to take home\\
    & & \\
    & $\left(\begin{array}{c}
     5\\
     3
\end{array}\right)$& $\left(\left(\begin{array}{c}
     5\\
     3
\end{array}\right)\right)=\left(\begin{array}{c}
     5+3-1\\
     3
\end{array}\right)$ \\\hline
        \multirow{5}{*}{Sensitive to order}& & \\
    &5 distinct books &unlimited copies of 5 books\\
    &give one to person A, B, and C &give one to person A, B, and C\\
    & & \\
    &$P(5,3)$ &$5^3$ \\
    \hline
\end{tabular}\caption{Segunda tabla de la tarea :D}
\end{center}
\label{Segunda tablal}
\end{table}\\
\begin{table}[h!]
\begin{center}
\begin{tabular}{|l||c|c|}
    \hline
     &No Repetiton&Repetition Allowed \\ \hline \hline
    \multirow{5}{*}{Not sensitive to order}& & \\
    &5 distinct books &unlimited copies of 5 books\\
    &choose 3 books to take home &choose 3 copies to take home\\
    & & \\
    & $\big(\begin{smallmatrix}
  5\\
  3 
\end{smallmatrix}\big)$ &$\big((\begin{smallmatrix}
  5\\
  3 
\end{smallmatrix}\big))=\big(\begin{smallmatrix}
  5+3-1\\
  3 
\end{smallmatrix}\big)$\\\hline
        \multirow{5}{*}{Sensitive to order}& & \\
    &5 distinct books &unlimited copies of 5 books\\
    &give one to person A, B, and C &give one to person A, B, and C\\
    & & \\
    &$P(5,3)$ &$5^3$ \\
    \hline
\end{tabular}\caption{Segunda tabla de la tarea :D}
\end{center}
\label{Segunda2 tablal}
\end{table}
\newpage
And here are some more formulas:\\
\\
$$\begin{pmatrix}
  n\\ 
  r 
\end{pmatrix}=\frac{n!}{(n-r)!r!}$$\newline
\\
$$\begin{pmatrix}
  n\\ 
  r 
\end{pmatrix}=\begin{pmatrix}
  n\\ 
  n-r 
\end{pmatrix}$$\newline
\\
$$\begin{pmatrix}
  n\\ 
  r 
\end{pmatrix}=\begin{pmatrix}
  n-1\\ 
  r-1 
\end{pmatrix}+\begin{pmatrix}
  n-1\\ 
  r 
\end{pmatrix}$$
Fin
\newpage
\section{Punto Tarea Revelo}
Reproduzca el documento a continuación.

\begin{center}
\textbf{Exercise 1: Sum rule for the Energy}
\end{center}
\begin{itemize}
\item Show that the expectation value of the total kinetic energy in the ground state of a many-body fermion system is given by
\begin{equation}
\langle T \rangle =-i\int d \vec{x} \lim_{\vec{x^{\prime}}\to\ \vec{x}} (-\frac{\hbar^{2}\nabla^{2}}{2m})\hspace{0.2cm} Tr[G(\vec{x},t;\vec{x^{\prime}},t^{+})]
\end{equation}
where we consider the trapping potential $U(\vec{x})=0$, and $\hat{T}=-\frac{\hbar^{2}\nabla^{2}}{2m}$.
\item Show that the total ground state energy is given by
\begin{eqnarray}
E=& \langle \hat{H} \rangle \nonumber \\
 =& \langle \hat{T} + \hat{V} \rangle \nonumber \\
 =& \langle \hat{T} \rangle + \langle \hat{V}  \nonumber \\
 =& -\frac{i}{2} \int d \vec{x} \lim_{\vec{x^{\prime}}\to\ \vec{x}} (i\hbar\frac{\partial}{\partial t}-\frac{\hbar^{2}\nabla^{2}}{2m})\hspace{0.2cm} Tr[G(\vec{x},t;\vec{x^{\prime}},t^{+})] 
\end{eqnarray}
\end{itemize}
\textit{Hint:} To evaluate the mean value of potential energy
\begin{equation}
\langle \hat{V} \rangle =\frac{\langle \psi_0 |\frac{1}{2} \int d \vec{x}
\int d \vec{x^{\prime}}\hat{\psi}_\beta^{\dagger}
(\vec{x}) \hat{\psi}_{\alpha}^{\dagger}(\vec{x^{\prime}}) V
(|\vec{x}-\vec{x^{\prime}}|)  \hat{\psi_\alpha }
(\vec{x^{\prime}}) \psi_\beta (\vec{x})|\psi_0\rangle}
{\langle\psi_0|\psi_0\rangle}
\end{equation}
use the result obtained in exercise 2, set 9:
\begin{equation}
\left[i\hbar\frac{\partial}{\partial t}+\frac{\hbar^{2}\nabla^{2}}{2m}\right]\hat{\psi}_{H\alpha} (\vec{x},t)=\int d\vec{x^{\prime}} \hat{\psi}_{H\beta}^{\dagger} (\vec{x}^{\prime},t)V(|\vec{x}-\vec{x}^{\prime}|\hat{\psi}_{H\beta}(\vec{x}^{\prime},t)\hat{\psi}_{H\alpha}(\vec{x},t)
\end{equation}
\section{Punto 5}
\begin{center}
{\small \textbf{DRIVLE’S THEOREM AND THE R-O LEMMA}}
\end{center}
In this Section we will state and prove our main result. The fundamental equation of wet fish-pricing is that of
Whackabath \cite{ref1}:
\begin{equation}\label{eqn1}
f_{xxx}+3f_{xx}-2\cdot Ker(f)=0,
\end{equation}
where Whackabath’s equation (\ref{eqn1}) is hardly ever used.\\
It is an interesting question whether the Whackabath’s equation (\ref{eqn1}) in standard topology can be applied without
change in Gackworth’s $\Omega$-topologies. A very full discussion of Gackworth’s work was given in \cite{ref1,ref2}.\\
$\cdots \cdots$
\renewcommand{\refname}{\centering \rule[0 mm]{30 mm} {0.1 mm}}
\begin{thebibliography}{99} %Bibliografia, información que tenemos en nuestras referencias %bibitem
\bibitem{ref1} T. I. Strainer \verb+&+ B. J. M. Wilkins 1993, A new result on Drivle’s Theorem, Proc. Iceland Cod Fish Soc. Lond. Ser. D, 134 (8678–8679).
\bibitem{ref2} B. J. M. Wilkins, “Topological Dynamics and the Haddock Fishery”, Unpublished, 1987.
\end{thebibliography}
\end{document}
