%Esto es un comentario :D 
%------------------Encabezado: tiene los paquetes----------------------------------
%Empezamos definiendo que tipo de documento vamos a usar.
%respetar la gerarquía entre corchetes y llaves. Paquetes y especifidad.

\documentclass[onecolumn]{article} %report, book %números de columnos
\usepackage[spanish]{babel} %Para las cosas en español (referencias, etc) Babel puede tener problemas.
%Renombrar los comandos.
\usepackage[latin1,utf8]{inputenc} %idioma, latin1:español, inputenc es el paquete de idiomas de occidente %utf8, guión, comas, mayor que, etc, lo que se puede teclar.
\usepackage{graphics,graphicx,xcolor} %xcolor, para colores %figuras
\usepackage{amssymb,amsmath} %Lenguaje matemático %amsmath, el de las integrales
\usepackage{verbatim} %Escribir código
\usepackage{bm} %negrita
%Cambiar margenes, usar más paquetes.
%Fecha en español. Nota: \renewcommand. También paquete para email.
%C y C++, Funciona de manera similar, con paquetes, primero los introducimos.
%Como escribir el simbolo porcentaje sin añadir comentario
%Reducir tamaño de margenes
%Cambiar fuente de letra, tamaño de letra, centrar, alinear a la izquierda, cambiar el temmplate. "centrar section latex"
%Como poner las comillas en español
%Cambiar simbolo de la viñeta

%---------------------------------------------------------------------
\title{Apuntes Phython}
\author{CARLOS ANDRES RODALLEGA MILLAN}
\date{\today}

%---------------------------------------------------------------------
%Toda pareja que comienza con \Begin y termina con \END se llama ENTORNO
%++++++++++++++++++++++CUERPO DEL DOCUMENTO+++++++++++++++++++++++++++
\begin{document}
\maketitle %HAGA el título.... (que está arriba)
%.-.-.-.-.
\section{Apuntes Phyton IMM} %Hace una sección
\subsection{Generalidades de los sistemas de cómputo}
Describir maneras para la realización de p-seudo códigos. 
Se habla del \textbf{Transistor} y sus implicaciones. Todo justificado desde la mecánica cuántica. Se hablan los elementos básicos de un computador. Los programas computacionales y los niveles de estos. Ambientes de desarrollo integrado. Desarrollo de algoritmos. 
\\
\textbf{Algoritmo}\\
Secuencia sistemática y finita de instrucciones que permiten realizar una determinada tarea. "Hojas de rutas" o "Diseños" de los programas. Antes de programar debemos tener el algoritmo de nuestro programa y también debemos tener en cuenta que el algoritmo no debe estar escrito en un lenguaje de programación necesariamente.



\section{Apuntes/Ejercicios libro personal} %Hace una sección


\end{document}
