% Derivatives
\renewcommand{\d}[0]{\mathrm{d}}
\newcommand{\dev}[2]{\displaystyle \frac{\d #1}{\d #2}}
\newcommand{\pdev}[2]{\displaystyle \frac{\partial #1}{\partial #2}}
\newcommand{\ndev}[3]{\displaystyle \frac{\d^{#3} #1}{\d #2^{#3} } }
\newcommand{\npdev}[3]{\displaystyle \frac{\partial^{#3} #1}{\partial #2^{#3} } }


%% Norms
\newcommand{\absvec}[1]{| \vec{#1} |}
\newcommand{\normvec}[1]{|\!| \vec{#1} |\!|}

\newcommand{\vmed}[1]{\left \langle #1 \right \rangle}
\newcommand{\vmedvec}[1]{\langle #1 \rangle}
\newcommand{\R}[0]{\mathbb{R}}
\renewcommand{\H}[0]{\operatorname{H}}

%Evidenziare formule
\newcommand{\mathcolorbox}[2]{\colorbox{#1}{$\displaystyle #2$}}
\newcommand{\hlfancy}[2]{\sethlcolor{#1}\hl{#2}}


%%%%%%%%%%%%%%%%%%%%Theorem, Corollary, Lemma, Proposition%%%%%%%%%%%%%%%%%
\usepackage[many,most,theorems]{tcolorbox}

\newtcbtheorem{theorem}{Theorem}{ % frame stuff
    boxrule = 1pt,
    breakable,
    enhanced,
    frame empty,
    interior style= {orange!20},
    %interior empty,
    colframe=black,
    borderline ={1pt}{0pt}{black},
    left=0.2cm,
    % title stuff
    attach boxed title to top left={yshift=-2mm,xshift=0mm},
    coltitle=black,
    fonttitle=\bfseries,
    colbacktitle=white,
    fontupper=\slshape,
    boxed title style={boxrule=1pt,sharp corners}}{theorem} 

\newtcbtheorem{corollary}{Corollary}{ % frame stuff
    boxrule = 1pt,
    breakable,
    enhanced,
    frame empty,
    interior style= {orange!20},
    %interior empty,
    colframe=black,
    borderline ={1pt}{0pt}{black},
    left=0.2cm,
    % title stuff
    attach boxed title to top left={yshift=-2mm,xshift=0mm},
    coltitle=black,
    fonttitle=\bfseries,
    colbacktitle=white,
    fontupper=\slshape,
    boxed title style={boxrule=1pt,sharp corners}}{corollary} 
    
\newtcbtheorem{lemma}{Lemma}{ % frame stuff
    boxrule = 1pt,
    breakable,
    enhanced,
    frame empty,
    interior style= {orange!20},
    %interior empty,
    colframe=black,
    borderline ={1pt}{0pt}{black},
    left=0.2cm,
    % title stuff
    attach boxed title to top left={yshift=-2mm,xshift=0mm},
    coltitle=black,
    fonttitle=\bfseries,
    colbacktitle=white,
    fontupper=\slshape,
    boxed title style={boxrule=1pt,sharp corners}}{lemma} 


%%%%%%%%%%%%%%%%%%%%Definition%%%%%%%%%%%%%%%%%


\newtcbtheorem{definition}{Definition}{ % frame stuff
    boxrule = 1pt,
    breakable,
    enhanced,
    frame empty,
    interior style= {blue!10},
    %interior empty,
    colframe=black,
    borderline ={1pt}{0pt}{black},
    left=0.2cm,
    % title stuff
    attach boxed title to top left={yshift=-2mm,xshift=0mm},
    coltitle=black,
    fonttitle=\bfseries,
    colbacktitle=white,
    boxed title style={boxrule=1pt,sharp corners}}{definition} 


%\theoremstyle{definition}
%\newtheorem{definition}{Definition}%[section]



%%%%%%%%%%%%%%%%%%%%Exercise and example%%%%%%%%%%%%%%%%%

\newtcbtheorem{exercise}{Exercise}{ % frame stuff
    boxrule = 1pt,
    breakable,
    enhanced,
    frame empty,
    interior style= {blue!6},
    %interior empty,
    colframe=black,
    borderline ={1pt}{0pt}{black},
    left=0.2cm,
    % title stuff
    attach boxed title to top left={yshift=-2mm,xshift=0mm},
    coltitle=black,
    fonttitle=\bfseries,
    colbacktitle=white,
    boxed title style={boxrule=1pt,sharp corners}}{exercise} 

\newtcbtheorem{example}{Example}{ % frame stuff
    boxrule = 1pt,
    enhanced,
    frame empty,
    interior style= {green!6},%{left color=yellow!70,right color=green!70},
    %interior empty,
    colframe=black,
    borderline ={1pt}{0pt}{black},
    breakable,
    left=0.2cm,
    % title stuff
    attach boxed title to top left={yshift=-2mm,xshift=0mm},
    coltitle=black,
    fonttitle=\bfseries,
    colbacktitle=white,
    boxed title style={boxrule=1pt,sharp corners}}{example}
  
%\newtheorem{exercise}{Exercise}
%\newtheorem{example}{Example}

%%%%%%%%%%%%%%%%%%%%%%%%%%%%%%%%%%%

\theoremstyle{remark}
\newtheorem*{remark}{Remark}
\newtheorem{observation}{Observation}
%Evidenziare testo
\newtheorem*{solution}{Solution}

\newcommand\mybox[1]{%
  \fbox{\begin{minipage}{0.9\textwidth}#1\end{minipage}}}

  %Spiegazioni/verifiche
\newenvironment{greenbox}{\begin{mdframed}[hidealllines=true,backgroundcolor=green!20,innerleftmargin=3pt,innerrightmargin=3pt,innertopmargin=3pt,innerbottommargin=3pt]}{\end{mdframed}}

\newenvironment{bluebox}{\begin{mdframed}[hidealllines=true,backgroundcolor=blue!10,innerleftmargin=3pt,innerrightmargin=3pt,innertopmargin=3pt,innerbottommargin=3pt]}{\end{mdframed}}

\newenvironment{yellowbox}{\begin{mdframed}[hidealllines=true,backgroundcolor=yellow!20,innerleftmargin=3pt,innerrightmargin=3pt,innertopmargin=3pt,innerbottommargin=3pt]}{\end{mdframed}}

\newenvironment{redbox}{\begin{mdframed}[hidealllines=true,backgroundcolor=red!20,innerleftmargin=3pt,innerrightmargin=3pt,innertopmargin=3pt,innerbottommargin=3pt]}{\end{mdframed}}

\newenvironment{orangebox}{\begin{mdframed}[hidealllines=true,backgroundcolor=orange!20,innerleftmargin=3pt,innerrightmargin=3pt,innertopmargin=3pt,innerbottommargin=3pt]}{\end{mdframed}}

%emph equation
\newcommand*\myyellowbox[1]{%
  \colorbox{yellow!40}{\hspace{1em}#1\hspace{1em}}}

\newcommand*\mygreenbox[1]{%
  \colorbox{green!20}{\hspace{1em}#1\hspace{1em}}}
  

%%%%%%%%%%PROOF%%%%%%%%%%%%%%%%%%%%%%%%%%%%%
\usepackage{xpatch}
\xpatchcmd{\proof}{\itshape}{\normalfont\proofnamefont}{}{}
\newcommand{\proofnamefont}{\bfseries}
\renewcommand\qedsymbol{$\blacksquare$}  
  



