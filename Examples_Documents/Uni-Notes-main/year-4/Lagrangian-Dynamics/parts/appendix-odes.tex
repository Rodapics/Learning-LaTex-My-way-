\chapter{Differential Equations}
Many of the equations of motion we will derive are second order linear ordinary differential equations of the form
\begin{equation}
    ay''(x) + by'(x) + cy(x) = f(x)
\end{equation}
for some constants \(a\), \(b\), and \(c\), and functions \(y\) and \(f\).
There is a general procedure for solving such equations.
\begin{enumerate}
    \item First define the homogenous equation
    \begin{equation}
        ay''(x) + by'(x) + cy(x) = 0.
    \end{equation}
    This is important since if \(y_1\) is a solution to the homogeneous equation and \(y_p\) is a solution to the inhomogeneous equation then \(y_p + \lambda y_1\) with \(\lambda\) being a constant, is a solution to the original inhomogeneous equation also.
    The most general solution is the solution to the inhomogeneous equation and a linear combination of solutions to the homogenous equation.
    The number of solutions to the homogenous equation is equal to the order of the equation, so for our purposes 2.
    \item Solve the homogenous equation.
    This is usually done by making a guess and substituting into the homogenous equation and solving for any unknowns in the guess.
    The most common guess is of the form \(\e^{mx}\).
    If we make this guess after substitution we have
    \begin{equation}
        (am^2 + bm + c)\e^{mx} = 0.
    \end{equation}
    Since \(\e^{mx} \ne 0\) we must therefore have that the quadratic in \(m\) is zero.
    This gives us two values for \(m\), \(m_1\) and \(m_2\).
    If \(m_1 \ne m_2\) then we define
    \begin{equation}
        y_1(x) = \e^{m_1x}, \qqand y_2(x) = \e^{m_2x}.
    \end{equation}
    If \(m_1 = m_2\) then instead define
    \begin{equation}
        y_1(x) = \e^{m_1x}, \qqand y_2(x) = x\e^{m_1x}.
    \end{equation}
    Occasionally other guesses, such as trig functions, make more sense, particularly if there are no first derivatives in the equation.
    
    \item Returning to the inhomogeneous equation we now need to find a solution, \(y_p\), of course if the original equation was inhomogeneous then this step can be skipped.
    We again do this by guessing.
    The exact guess depends on the form of \(f\) but some common choices are given in \cref{tab:common guesses for differential equation}.
    \begin{table}
        \caption{Common guesses to solve the inhomogeneous differential equation \(ay''(x) + by'(x) + cy(x) = f(x)\) for different functions \(f\).}
        \label{tab:common guesses for differential equation}
        \begin{tabular}{cc}
            \toprule
            Form of \(f\) & Suggestion for \(y_p\)\\\midrule
            Constant, \(k\) & Constant \(k/c\)\\
            Polynomial in \(x\) & Polynomial in \(x\)\\
            \(ke^{\alpha x}\) & \(A e^{\alpha x}\)\\
            \(\sin(\alpha x)\) and/or \(\cos(\alpha x)\) & \(A\sin(\alpha x) + B\cos(\alpha x)\)\\\bottomrule
        \end{tabular}
    \end{table}
    Once you have chosen a function substitute it in and see if it is valid.
    
    \item If you have no boundary conditions then you are done.
    The solution is
    \begin{equation}
        y(x) = y_p(x) + \sum_i\lambda_i y_i(x)
    \end{equation}
    where \(y_i\) are solutions to the homogeneous equation.
    If you have boundary conditions then substitute them into this equation and solve for \(\lambda_i\).
\end{enumerate}

\begin{exm}{}{}
    Suppose we wish to solve
    \begin{equation}
        y''(x) + 4y'(x) + 4y(x) = \sin(3x),
    \end{equation}
    with the boundary conditions \(y(0) = 10\) and \(y'(0) = 0\).
    
    We start by defining the homogeneous equation
    \begin{equation}
        y''(x) + 4y'(x) + 4y(x) = 0.
    \end{equation}
    We propose that \(\e^{mx}\) is a solution to this.
    Substituting this in we find
    \begin{equation}
        (m^2 + 4m^2 + 4)\e^{mx} = 0.
    \end{equation}
    The quadratic has one repeated root \(m = -2\) and so our solutions to the homogenous equation are
    \begin{equation}
        y_1(x) = \e^{-2x}, \qqand y_2(x) = x\e^{-2x}.
    \end{equation}
    
    In this case \(f(x) = \sin(3x)\) and so we try a solution of the form
    \begin{equation}
        y_p(x) = A\sin(3x) + B\cos(3x).
    \end{equation}
    Substituting this into our original equation we have
    \begin{multline}
        -9A\sin(3x) -9B\cos(3x) + 12A\cos(3x) - 12B\sin(3x)\\ + 4A\sin(3x) + 4B\cos(3x)\\= -(5A + 12B)\sin(3x) + (12A - 5B)\cos(3x).
    \end{multline}
    In order for this to be equal to \(\sin(3x)\) we require that
    \begin{equation}
        5A + 12B = -1, \qqand 12A - 5B = 0.
    \end{equation}
    Solving these simultaneous equations gives us
    \begin{equation}
        A = -\frac{5}{169}, \qqand B = -\frac{12}{169}.
    \end{equation}
    Hence our solution is
    \begin{equation}
        y_p(x) = -\frac{5}{169} \sin(3x) - \frac{12}{169} \cos(3x).
    \end{equation}
    
    The general solution is then
    \begin{equation}
        y(x) = -\frac{5}{169} \sin(3x) - \frac{12}{169} \cos(3x) + \lambda_1\e^{-2x} + \lambda_2x\e^{-2x}.
    \end{equation}
    Substituting for our first initial condition, \(y(0) = 10\), we have
    \begin{equation}
        y(0) = -\frac{12}{169} + \lambda_1 = 10 \implies \lambda_1 = \frac{1702}{169}.
    \end{equation}
    The second initial condition, \(y'(0) = 0\), gives us
    \begin{equation}
        y'(0) = -\frac{15}{169} - 2\frac{1702}{169} - 2\lambda_2 = 0 \implies \lambda_2 = -\frac{263}{26}.
    \end{equation}
    Hence our final solution is
    \begin{equation}
        y(x) = -\frac{5}{169} \sin(3x) - \frac{12}{169} \cos(3x) + \frac{1702}{169}\e^{-2x} - \frac{263}{26}x\e^{-2x}.
    \end{equation}
\end{exm}
