\chapter{Fermi's Golden Rule}
\label{app:fermi's golden rule}
\begin{rmk}
    This derivation is repeated, in varying levels of detail with differing formalisms, in \course{Principles of Quantum Mechanics} and \course{Quantum Theory}.
\end{rmk}
\section{Derivation to First Order}
Consider the Hamiltonian
\begin{equation}
    \operator{H} = \operator{H}_0 + \operator{H}'(t, \vv{x})
\end{equation}
where \(\operator{H}_0\) is a time independent Hamiltonian for which we can solve the Schrödinger equation and \(\operator{H}'(t, \vv{x})\) is the interaction Hamiltonian.
Let \(\varphi_k(t, \vv{x})\) be a normalised solution to the Schrödinger equation for the unperturbed Hamiltonian, that is
\begin{equation}
    \operator{H}_0\varphi_k = E_k\varphi_k, \qqand \braket{j}{k} = \delta_{jk}
\end{equation}
where \(\varphi_k(\vv{x}) = \braket{\vv{x}}{k}\).

The Schrödinger equation in the presence of the interaction Hamiltonian is
\begin{equation}
    i\diffp{\psi}{t} = [\operator{H}_0 + \operator{H}'(t, \vv{x})]\psi.
\end{equation}
The wave function, \(\psi\), can be expressed in terms of the complete set of states of the eigenstates of the unperturbed Hamiltonian:
\begin{equation}
    \psi(t, \vv{x}) = \sum_k c_k(t) \varphi_k(\vv{x}) \e^{-iE_kt}.
\end{equation}
Substituting this into the Schrödinger equation we get
\begin{equation}
    i \sum_k \left[ \diff{c_k}{t} \varphi_k \e^{-iE_kt} - iE_kc_k\varphi_k\e^{-iE_kt} \right] = \sum_k \left[ c_k\operator{H}_0\varphi_k \e^{-iE_kt} + \operator{H}'c_k\varphi_k\e^{-iE_kt} \right]
\end{equation}
Using \(\operator{H}_0\varphi_k = E_k\varphi_k\) the second term on the left cancels with the first on the right and we're left with
\begin{equation}\label{eqn:fermi's golden rule DE for cf}
    i \sum_k \diff{c_k}{t} \varphi_k\e^{-iE_kt} = \sum_k \operator{H}'c_k(t) \varphi_k \e^{-iE_kt}.
\end{equation}

Suppose that at time \(t = 0\) the initial state is \(\varphi_i\), and the coefficients are \(c_k(0) = \delta_{ik}\), that is \(c_k(0) = 0\) for all \(k \ne i\).
Suppose also that the perturbing Hamiltonian is constant for \(t > 0\), we can imagine switching it on at time \(t = 0\) and then leaving it on, and that its small enough that at all times \(c_i(t) \approx 1\) and \(c_k(t) \approx 0\) for \(k \ne i\).
Then, we have
\begin{equation}
    i \sum_k \diff{c_k}{t} \varphi_k \e^{-iE_kt} \approx \operator{H}'\varphi_i\e^{-iE_it}.
\end{equation}
Rewriting this in terms of kets we have
\begin{equation}
    i \sum_k \diff{c_k}{t} \e^{-iE_kt} \ket{k} \approx \e^{-iE_it}\operator{H}'\ket{i}.
\end{equation}
Taking the product with some particular final state, \(\bra{f}\), we have
\begin{equation}
    i \sum_k \diff{c_k}{t} \e^{-iE_kt} \braket{\varphi_f}{\varphi_i} = i\diff{c_f}{t} \e^{-iE_ft} \approx \e^{-iE_it} \bra{f} \operator{H}' \ket{i}.
\end{equation}
Rearranging this we get
\begin{equation}
    \diff{c_f}{t} = -i \bra{f} \operator{H}' \ket{i} \e^{i(E_f - E_i)t}.
\end{equation}
Here we use the usual inner product
\begin{equation}
    \bra{f} \operator{H}' \ket{i} = \int \varphi_f^*(\vv{x}) \operator{H}' \varphi_i(\vv{x}) \dd{^3\vv{x}}.
\end{equation}

Define the transition matrix element, \(T_{fi} = \bra{f} \operator{H}' {i}\).
At time \(t\) the amplitude for transitions to the state \(\ket{f}\) is given by
\begin{equation}
    c_f(t) = -i \int_0^t T_{fi} \e^{i(E_f - E_i)t'} \dd{t'}.
\end{equation}
If the perturbing Hamiltonian is time independent then so is \(\bra{f} \operator{H}' \ket{i}\), and so
\begin{equation}\label{eqn:fermi's golden rule cf}
    c_f(t) = -iT_{fi} \int_0^t \e^{i(E_f - E_i)t'} \dd{t'}.
\end{equation}
The probability to transition to the state \(\ket{f}\) if we start in the state \(\ket{i}\) is then
\begin{equation}
    \probability(f \to i) = c_f^*(t)c_f(t) = \abs{T_{fi}}^2 \int_{0}^{t} \int_{0}^{t} \e^{i(E_f - E_i)t'} \e^{-i(E_f - E_i)t''} \dd{t'} \dd{t''}.
\end{equation}

The transition rate, \(\dl{\Gamma_{fi}}\), to transition from the given initial state to the particular final state, \(\ket{f}\), is then
\begin{equation}
    \dl{\Gamma_{fi}} = \frac{1}{t} \probability(f \to i) = \frac{1}{t} \abs{T_{fi}}^2 \int_0^t \int_0^t \e^{i(E_f - E_i)t'} \e^{-i(E_f - E_i)t''} \dd{t'} \dd{t''}.
\end{equation}
We can make the substitution \(t' \to t' + t/2\) and \(t'' \to t'' + t/2\).
This shifts the integration limits, without changing the integrand, since
\begin{equation}
    \e^{kt'}\e^{-kt''} \to \e^{kt' + kt/2}\e^{-kt'' - kt/2} = \e^{kt'}\e^{kt''},
\end{equation}
with \(k = i(E_f - E_i)\).

Performing this integral we get
\begin{equation}
    4\sinc^2\left( \frac{t}{2}[E_f - E_i] \right).
\end{equation}
\begin{cde}{}{}
    \begin{lstlisting}[gobble=8, language=mathematica, mathescape]
        Integrate[
        Integrate[
        Exp[I k t1] Exp[-I k t2],
        {t1, -t/2, t/2}],
        {t2, -t/2, t/2},
        Assumptions -> {k $\in$ Reals, t $\in$ Reals}]
        4 Sin[kt/2]^2/k^2
    \end{lstlisting}
\end{cde}
It is well known\footnote{see \course{Methods of Mathematical Physics}} that \(\sinc^2 x\) approximates \(\delta(x)\).
This means that the transition rate is only significant to states with \(E_f \approx E_i\).
This is energy conservation within the limits of the uncertainty relation, \(\Delta E \, \Delta t \approx 1/2\).
We use this to symmetrically extend the limits to \(\pm \infty\):
\begin{equation}
    \dl{\Gamma_{fi}} = \abs{T_{fi}}^2 \lim_{t \to \infty} \left[ \frac{1}{t} \int_{-t/2}^{t/2} \int_{-t/2}^{t/2} \e^{i(E_f - E_i)t'} \e^{-i(E_f - E_i)t''} \dd{t'} \dd{t''} \right]
\end{equation}

We can then recognise the integral representation of the Dirac delta:
\begin{equation}
    \delta(x) = \frac{1}{2\pi} \int_{-\infty}^{\infty} \int_{-\infty}^{\infty} \e^{ipx} \dd{p}.
\end{equation}
This comes from realising that the Fourier transform of \(\delta(x)\) is
\begin{equation}
    \fourierTransform \{\delta(x)\} = \int_{-\infty}^{\infty} \delta(x) \e^{-ipx} \dd{x} = \e^{0} = 1,
\end{equation}
and then the above expression for \(\delta(x)\) is simply \(\inverseFourierTransform\{\fourierTransform\{\delta(x)\}\}\).

We can use this to write
\begin{equation}
    \dl{\Gamma_{fi}} = 2\pi\abs{T_{fi}}^2 \lim_{t \to \infty} \left[ \frac{1}{t} \int_{-t/2}^{t/2} \e^{i(E_f - E_i)t'} \delta(E_f - E_i) \dd{t'} \right].
\end{equation}
Suppose there are \(\dl{n}\) accessible final states in the range \([E_f, E_f + \dl{E_f}]\).
Then the total transition rate is given by integrating over the final states with accessible energy, which we convert to an integral over their energies using the density of states\footnote{see the \course{Statistical Mechanics} part of the \course{Thermal Physics} course.}:
\begin{align}
    \Gamma_{fi} &= 2\pi \int \abs{T_{fi}}^2 \diff{n}{E_f} \delta(E_f - E_i) \lim_{t \to \infty} \left[ \frac{1}{t} \int_{-t/2}^{t/2} \dl{t} \right] \dd{E_f}\\
    &= 2\pi \int \abs{T_{fi}} \diff{n}{E_f} \delta(E_f - E_i) \dd{E_f}\\
    &= 2\pi \abs{T_{fi}} \diff{n}{E_f}[E_i].
\end{align}
The last term here is the density of states:
\begin{equation}
    \rho(E_i) = \diff{n}{E_f}[E_i].
\end{equation}
Thus, we have derived Fermi's golden rule
\begin{equation}
    \Gamma_{fi} = 2\pi \abs{T_{fi}}^2 \rho(E_i).
\end{equation}

\section{Improvement to Second Order}
To first order we have
\begin{equation}
    T_{fi} = \bra{f} \operator{H}' \ket{i}.
\end{equation}
We assumed that \(c_k(t) \approx 0\) for \(k \ne i\).
An improved derivation would again take \(c_i(t) \approx 1\) and substitute the expression for \(c_k(t)\) from \cref{eqn:fermi's golden rule cf} and substitute it into \cref{eqn:fermi's golden rule DE for cf}.
Then again taking the inner product with \(\ket{f}\) we get
\begin{equation}
    \diff{c_f}{t} \approx -i \bra{f} \operator{H} \ket{i} \e^{i(E_f - E_i)t} + (-i)^2 \sum_{k \ne i} \bra{f} \operator{H}'\ket{k} \e^{i(E_f - E_k)t} \int_{0}^{t} \bra{k} \operator{H}' \ket{i} \e^{i(E_k - E_i)t'} \dd{t'}.
\end{equation}

The perturbation is not present at time \(t = 0\) and for \(t > 0\) the perturbation is constant, which lets us write
\begin{equation}
    \int_0^t \bra{k} \operator{H}' \ket{i} \e^{i(E_k - E_i)t'} \dd{t'} = \bra{k} \operator{H}'\ket{i} \frac{\e^{i(E_k-E_i)t}}{i(E_k - E_i)}.
\end{equation}
This gives us an improved approximate differential equation for the coefficients \(c_f(t)\):
\begin{equation}
    \diff{c_f}{t} \approx -i \left[ \bra{f} \operator{H}' \ket{i} + \sum_{k \ne i} \frac{\bra{f} \operator{H}' \ket{k} \bra{k} \operator{H}' \ket{i}}{E_i - E_k} \right] \e^{i(E_f - E_i)t}.
\end{equation}
Then, to second order, the transition matrix elements are
\begin{equation}
    T_{fi} = \bra{f} \operator{H}' \ket{i} + \sum_{k \ne i} \frac{\bra{f} \operator{H}' \ket{k} \bra{k} \operator{H}' \ket{i}}{E_i - E_k}.
\end{equation}
We interpret the first term as a direct transition, \(\ket{i} \to \ket{f}\).
The second term then corresponds to some indirect transition, \(\ket{i} \to \ket{k} \to \ket{f}\), where we sum over all possible intermediate states, \(\ket{k}\).
The next higher order term would then allow two intermediate states, and so on.