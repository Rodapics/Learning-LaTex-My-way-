\chapter{Gauge Boson Polarisations}
\section{Classical Electrodynamic Gauge Invariance}
Maxwell's equations are given by
\begin{equation}
    \partial_\mu F^{\mu\nu} = j^\nu \qqand \partial_\mu F^{*\mu\nu} = 0
\end{equation}
where
\begin{equation}
    F^{\mu\nu} \coloneqq \partial^\mu A^\nu - \partial^\nu A^\mu
\end{equation}
is the electric field strength tensor, \(A^\mu = (\varphi, \vv{A})\) is the four-potential, \(j^\nu = (\rho, \vv{j})\) is the four-current, and
\begin{equation}
    F^{*\mu\nu} = \frac{1}{2}\varepsilon^{\mu\nu\rho\sigma}F_{\rho\sigma}
\end{equation}
is the dual field strength tensor.

The first equation can be expanded in terms of \(A\) to get
\begin{equation}
    \partial_\mu\partial^\mu A^\nu - \partial_\mu\partial^\nu A^\mu = \dalembertian A^\nu - \partial^\nu\partial_\mu A^\mu = j^\nu.
\end{equation}

We know that in electromagnetism we have a gauge freedom where the transformation
\begin{equation}
    \varphi \mapsto \varphi' - \diffp{\chi}{t}, \qqand \vv{A} \mapsto \vv{A} + \grad \chi
\end{equation}
for some function \(\chi\) with continuous second derivatives doesn't change Maxwell's equations.
This can easily be shown by substituting these transformed potentials into Maxwell's equations, through \(\vv{B} = \curl (\vv{A} + \grad\chi)\) and \(\vv{E} = -\grad\varphi -\partial_t(\vv{A} + \grad\chi)\) and then using the identities \(\div(\curl\vv{X}) = 0\) and \(\curl(\grad f) = 0\).

Relativistically this gives the gauge freedom
\begin{equation}
    A_\mu \mapsto A_\mu - \partial_\mu \chi.
\end{equation}
Under this we have
\begin{equation}
    \partial_\mu A^\mu \mapsto \partial_\mu A'^\mu = \partial_\mu (A^\mu - \partial^\mu \chi) = \partial_\mu A^\mu - \dalembertian \chi.
\end{equation}
We can choose \(\dalembertian\chi = \partial_\mu A^\mu\), then our transformed potential has \(\partial_\mu A'^\mu = 0\).
This is called the \defineindex{Lorenz gauge} condition.
In this gauge Maxwell's equation becomes
\begin{equation}
    \dalembertian A^\mu = j^\mu.
\end{equation}


\section{Photon Polarisations}
In a vacuum, \(j^\nu = 0\), the photon field satisfies \(\dalembertian A^\mu = 0\).
This has plane wave solutions:
\begin{equation}
    A^\mu = \varepsilon^\mu(q) \e^{-iq\cdot x}
\end{equation}
where \(q\) is the four-momentum of the photon and \(\varepsilon^\mu(p)\) describes the polarisation of the electromagnetic field.
Substituting this into the equation of motion we get
\begin{equation}
    \dalembertian A^\mu = -q^2\varepsilon^\mu(q)\e^{-iq\cdot x} = 0.
\end{equation}
Hence the plane wave solutions must have \(q^2 = 0\) for a free photon field.

A spin 1 boson, such as the photon, has three degrees of freedom, spins \(-1\), \(0\), and \(+1\).
It's not immediately obvious how these correspond to the four degrees of freedom in defining \(\varepsilon^\mu\).
The solution is that gauge freedom fixes one of the degrees of freedom.
In the Lorenz gauge we must have \(\partial_\mu A^\mu = 0\), and so
\begin{equation}
    0 = \partial_\mu (\varepsilon^\mu \e^{-iq\cdot x}) = -iq_\mu\varepsilon^\mu\e^{-iq \cdot x}.
\end{equation}
We must therefore have
\begin{equation}
    q_\mu \varepsilon^\mu = 0,
\end{equation}
fixing one degree of freedom in \(\varepsilon^\mu\).

There is still further freedom to make the gauge transformation
\begin{equation}
    A_\mu \mapsto A_\mu - \partial_\mu \Lambda(x)
\end{equation}
where \(\Lambda\) is some function satisfying \(\dalembertian\Lambda = 0\).
Consider the gauge transformation defined by \(\Lambda = -ia\e^{-iq \cdot x}\), which satisfies \(\dalembertian\Lambda = -q^2\Lambda = 0\), since \(q^2 = 0\).
Under this transformation \(A_\mu\) becomes
\begin{align}
    A_\mu \mapsto A'_\mu &= A_\mu - \partial_\mu\Lambda\\
    &= \varepsilon_\mu \e^{-iq\cdot x} + ia\partial_\mu\e^{-iq\cdot x}\\
    &= \varepsilon_\mu \e^{-iq\cdot x} + ia(-iq_\mu)\e^{-iq\cdot x}\\
    &= (\varepsilon_\mu + aq_\mu)\e^{-iq\cdot x}.
\end{align}
We can then interpret this gauge freedom as the ability to make the transformation
\begin{equation}
    \varepsilon_\mu \mapsto \varepsilon_\mu + aq_\mu
\end{equation}
without changing the physics.
The \defineindex{Coulomb gauge} corresponds to the choice of \(a\) making the time component of the polarisation, \(\varepsilon_0\), vanish.
The Lorenz gauge condition, \(\varepsilon_\mu q^\mu\), then becomes \(\vv{\varepsilon} \cdot \vv{q} = 0\).

These requirements are satisfied by
\begin{equation}
    \varepsilon^{(1)} = 
    \begin{pmatrix}
        0\\ 1\\ 0\\ 0
    \end{pmatrix}
    , \qqand \varepsilon^{(2)} = 
    \begin{pmatrix}
        0\\ 0\\ 1\\ 0
    \end{pmatrix}
    .
\end{equation}

\section{Polarisation of Massive Spin-1 Particles}
A massless noninteracting spin 1 field has the Lagrangian\footnote{see \course{Classical Electrodynamics} for a derivation}
\begin{equation}
    \lagrangianDensity_0 = -\frac{1}{4}F^{\mu\nu}F_{\mu\nu}.
\end{equation}
Here \(F^{\mu\nu}\) is the field-strength tensor, given by \(F^{\mu\nu} = \partial^\mu B^\nu - \partial^\nu B^\mu\), where we now use \(B\) instead of \(A\) because we no longer want to focus only on photons.
For massive particles we add in a mass term\footnote{see \course{Quantum Field Theory} for more details}:
\begin{equation}
    \lagrangianDensity_m = -\frac{1}{4}F^{\mu\nu}F_{\mu\nu} + \frac{1}{2}m^2B^\mu B_\mu.
\end{equation}
The Euler--Lagrange equations then give\footnote{see \course{Classical Electrodynamics} for a detailed calculation of \(\diffp{F^{\mu\nu}F_{\mu\nu}}/{(\partial_\mu B^\nu)}\).}
\begin{equation}
    (\dalembertian + m^2)B^\mu - \partial^\mu\partial_\nu B^\nu = 0.
\end{equation}

An alternative way to obtain this equation is to note that for a massless scalar particle the Klein--Gordon equation is \(\dalembertian\varphi = 0\), whereas for a massive scalar field it's \((\dalembertian + m^2)\varphi = 0\), suggesting that applying the prescription \(\dalembertian \mapsto \dalembertian + m^2\) to a massless field equation gives the equivalent equation for a massive particle.
Which we can then apply to the massless photon free equation
\begin{equation}
    \dalembertian A^\mu - \partial^\mu\partial_\nu A^\nu = 0
\end{equation}
to get the result.

Taking the derivative of the resulting equation of motion we get
\begin{equation}
    0 = (\dalembertian + m^2) \partial_\mu B^\mu - \partial_\mu\partial^\mu\partial_\nu B^\nu = (\dalembertian + m^2) \partial_\mu B^\mu - \dalembertian \partial_\mu B^\mu = m^2\partial_\mu B^\mu = 0.
\end{equation}
This implies that massive spin 1 particles automatically satisfy the Lorenz condition,
\begin{equation}
    \partial_\mu B^\mu = 0.
\end{equation}
Using this the equation of motion becomes
\begin{equation}
    (\dalembertian + m^2) B^\mu = 0.
\end{equation}
So we just get a Klein--Gordon equation for each component.

A massive particle with four-momentum \(q\) has \(q^2 = m^3\) and so
\begin{equation}
    \dalembertian \e^{-iq\cdot x} = -q^2\e^{-q\cdot x} = -m^2\e^{-iq\cdot x}
\end{equation}
implying plane wave solutions of the form
\begin{equation}
    B^\mu = \varepsilon^\mu \e^{-iq\cdot x}.
\end{equation}
The Lorenz condition implies that
\begin{equation}
    q_\mu \varepsilon^\mu = 0,
\end{equation}
exactly the same as in the massless case.

There is no further gauge freedom, there isn't a massive equivalent of the Coulomb gauge, and so we have three different polarisations, we can take the two used for the photon and as the third
\begin{equation}
    \varepsilon^{(3)} = 
    \begin{pmatrix}
        0\\ 0\\ 0\\ 1
    \end{pmatrix}
    .
\end{equation}

\section{Polarisation Sums}
A polarisation sum is a sum of the form
\begin{equation}
    \sum_\lambda \varepsilon^{(\lambda)}_\mu (\varepsilon^{(\lambda)}_\nu)^*.
\end{equation}

\subsection{Massive Gauge Bosons}
We can interpret \(\varepsilon^{(\lambda)}_\mu(\varepsilon^{(\lambda)}_\nu)^*\) as the components of some two-index tensor, or a matrix.
Rather than work with the polarisation vectors we've discussed, instead we use the left and right circular polarisations
\begin{equation}
    \varepsilon^{(-)} = \frac{1}{\sqrt{2}}
    \begin{pmatrix}
        0\\ 1\\ -i\\ 0
    \end{pmatrix}
    , \qand 
    \varepsilon^{(+)} = -\frac{1}{\sqrt{2}}
    \begin{pmatrix}
        0\\ 1\\ i\\ 0
    \end{pmatrix}
    .
\end{equation}
As the third polarisation state we then choose something orthogonal to both of these, so it must be of the form \((\alpha, 0, 0, \beta)^\trans\).
The relationship between \(\alpha\) and \(\beta\) is fixed by the requirement that \(q_\mu \varepsilon^\mu = 0\), so we must have \(\alpha E - \beta p_z = 0\).
Hence we choose the longitudinal polarisation to be
\begin{equation}
    \varepsilon^{(\symup{L})} = \frac{1}{m}
    \begin{pmatrix}
        p_z\\ 0\\ 0\\ E
    \end{pmatrix}
    ,
\end{equation}
where the normalisation is chosen such that in the rest frame we just have \((0, 0, 0, 1)^\trans\).

Using this our polarisation sum becomes
\begin{align}
    \sum_\lambda \varepsilon_\mu^{(\lambda)} (\varepsilon_\nu^{(\lambda)})^* &= \varepsilon_\mu^{(+)}(\varepsilon_\nu^{(+)})^* + \varepsilon_\mu^{(-)}(\varepsilon_\nu^{(-)})^* + \varepsilon_\mu^{(\symup{L})}(\varepsilon_\nu^{(\symup{L})})^*\\
    &= \frac{1}{2}
    \begin{pmatrix}
        0 & 0 & 0 & 0\\
        0 & 1 & -i & 0\\
        0 & -i & 1 & 0\\
        0 & 0 & 0 & 0
    \end{pmatrix}
    + \frac{1}{2}
    \begin{pmatrix}
        0 & 0 & 0 & 0\\
        0 & 1 & -i & 0\\
        0 & i & 1 & 0\\
        0 & 0 & 0 & 0
    \end{pmatrix}
    \\
    &\qquad\qquad+ \frac{1}{m^2}
    \begin{pmatrix}
        p_z^2 & 0 & 0 & Ep_z\\
        0 & 0 & 0 & 0\\
        0 & 0 & 0 & 0\\
        Ep_z & 0 & 0 & E^2
    \end{pmatrix}
    \\
    &= 
    \begin{pmatrix}
        -1 & 0 & 0 & 0\\
        0 & 1 & 0 & 0\\
        0 & 0 & 1 & 0\\
        0 & 0 & 0 & 0
    \end{pmatrix}
    + \frac{1}{m^2}
    \begin{pmatrix}
        (q^0)^2 & 0 & 0 & q^0q^3\\
        0 & 0 & 0 & 0\\
        0 & 0 & 0 & 0\\
        q^0q^3 & 0 & 0 & (q^3)^2
    \end{pmatrix}
    .
\end{align}
In the last step we consider that the particle is moving in the \(z\)-direction, by definition, and wrote \(E^2 = m^2 + p_z^2\), so \(p_z^2/m^2 = (E^2 - m^2)/m^2 = E^2 - 1\), and similarly \(E^2/m^2 = (m^2 + p_z^2)/m^2 = 1 + p_z^2\), and then we wrote \(p_z\) and \(E\) in terms of components of the four-momentum, \(q\).
This generalises to a particle moving along an arbitrary direction:
\begin{equation}
    \sum_\lambda \varepsilon_\mu^{(\lambda)}(\varepsilon_\nu^{(\lambda)})^* = -\minkowskiMetric_{\mu\nu} + \frac{q_\mu q_\nu}{m^2}.
\end{equation}

\subsection{Real Photons}
For photons the gauge freedom makes matters slightly more complicated.
For a real photon travelling in the \(z\)-direction we can use the polarisations \(\varepsilon^{(1)} = (0, 1, 0, 0)^\trans\) and \(\varepsilon^{(2)} = (0, 0, 1, 0)^\trans\), and these are the only polarisation states.
Hence,
\begin{equation}
    \sum_\lambda \varepsilon_\mu^{(\lambda)}(\varepsilon_\nu^{(\lambda)})^* = 
    \begin{pmatrix}
        0 & 0 & 0 & 0\\
        0 & 1 & 0 & 0\\
        0 & 0 & 1 & 0\\
        0 & 0 & 0 & 1
    \end{pmatrix}
    .
\end{equation}
As before this generalises to a particle travelling in an arbitrary direction:
\begin{equation}
    \sum_T \varepsilon_i^{(T)}(\varepsilon_j^{(T)})^* = \delta_{ij} - \frac{q_iq_j}{\abs{\vv{q}}^2}.
\end{equation}
Here the sum is over the transverse polarisation states, indexed by \(T\).

If we want to sum over all possible components of the polarisation vector then we need to account for extra gauge freedom.
To do so consider the process \(\Pq \to \Pq \Pphoton\), such as occurs as part of the decay \(\Prhozero \to \Ppizero \Pphoton\).
This is shown in the diagram
\begin{equation}
    \tikzsetnextfilename{fd-quark-to-quark-photon}
    \feynmandiagram[horizontal'=i to v, inline=(v)]{
        i [particle=\Pq] -- [fermion,  edge label'=\(p\)] v -- [fermion, edge label=\(p'\)] o1 [particle=\Pq],
        v -- [photon, edge label=\(q\)] o2 [particle=\Pphoton]
    };
\end{equation}
The amplitude for this vertex is
\begin{equation}
    \amplitude = Q_{\Pq} \diracadjoint{u}(p')\gamma^\mu u(p) \varepsilon_\mu^{(\lambda)}(q),
\end{equation}
which follows in a similar way to \cref{eqn:electron-photon interaction strength}.
We can write this as
\begin{equation}
    \amplitude = j^\mu \varepsilon^{(\lambda)}_\mu(q), \qqwhere j^\mu = Q_{\Pq} \diracadjoint{u}(p')\gamma^\mu u(p).
\end{equation}
Then, summing over all polarisations, we have
\begin{equation}
    \sum_\lambda \abs{\amplitude}^2 = \sum_\lambda j^\mu (j^\nu)* \varepsilon^{(\lambda)}_\mu (\varepsilon_\nu^{(\lambda)})^*.
\end{equation}
In the Coulomb gauge this takes the form
\begin{equation}
    \sum_{T = 1}^2 \abs{\amplitude}^2 = j^\mu(j^\nu)^* \sum_{T = 1}^2 \varepsilon_\mu^{(T)}(\varepsilon_\nu^{(T)})^*.
\end{equation}
In the frame where the photon travels along the \(z\)-direction this becomes
\begin{equation}
    \sum_{T = 1}^{2} \abs{\amplitude}^2 = j_1j_1^* + j_2j_2^*.
\end{equation}
We can write this as a sum over all four components by subtracting off the unwanted components:
\begin{equation}
    \sum_{T = 1}^{2} \abs{\amplitude}^2 = j_1j_1^* + j_2j_2^* = -\minkowskiMetric^{\mu\nu}j_\mu j_\nu^* + j_0j_0^* - j_3j_3^*.
\end{equation}

We saw that the polarisation vectors \(\varepsilon_\mu\) and \(\varepsilon_{\mu} + aq^\mu\) describe the same physics, so the amplitude must be invariant under this, meaning we must have
\begin{equation}
    \amplitude = j^\mu \varepsilon_\mu^{(\lambda)}(q) = j^\mu \varepsilon^{(\lambda)}_\mu(q) + j^\mu q_\mu.
\end{equation}
Hence, we must have
\begin{equation}
    j_\mu q^\mu = 0.
\end{equation}
For a photon with \(q^\mu = (q, 0, 0, q)\) (so \(q^2 = 0\)) we have \(qj_0 - qj_3 = 0\), and so the invariance condition gives \(j_0 = j_3\).
Hence, 
\begin{equation}
    \sum_{T = 1}^{2} \abs{\amplitude}^2 = -\minkowskiMetric^{\mu\nu}j_\mu j_\nu.
\end{equation}
Thus, the sum over polarisation states for a real photon gives
\begin{equation}
    \sum_{T = 1}^{2} \varepsilon^{(\lambda)}_{\mu} (\varepsilon^{(\lambda)}_\nu)^* = -\minkowskiMetric_{\mu\nu}.
\end{equation}

\subsection{Virtual Photons}
For off-shell photons \(q^2 \ne 0\) and we cannot ignore the other two polarisation states.
The amplitude for \(\Pe\Ptau \to \Pe\Ptau\) scattering via a photon in a \(t\)-channel process is, as given in \cref{eqn:electron tau scattering amplitude},
\begin{equation}
    \amplitude \propto \sum_{\lambda = 1}^{4} j_\mu^{(\Pe)} j_\nu^{(\Ptau)} \frac{\varepsilon^{(\lambda)}_\mu(\varepsilon^{(\lambda)}_\nu)^*}{q^2}.
\end{equation}
We can treat the virtual photon as having effective mass \(m^2 = q^2\), and so we can use the massive spin 1 result:
\begin{equation}
    \sum_{\lambda = 1}^{4} \varepsilon^{(\lambda)}_\mu(\varepsilon^{(\lambda)}_\nu)^* = -\minkowskiMetric_{\mu\nu} + \frac{q_\mu q_\nu}{q^2}.
\end{equation}
Now, the \(t\)-channel process is the only one at this order in in QED.
There is no \(s\) channel, as we'd need a boson with charge \(-2\) to propagate between the two vertices.
There is no \(u\) channel, since the two particles are distinct.
This means that this term must be gauge invariant on its own.
So, consider the gauge transformation \(\varepsilon^\mu \mapsto \varepsilon^\mu + aq^\mu\) and \(\varepsilon^\nu \mapsto \varepsilon^\nu + bq^\nu\), we can take different gauge transformations at different points in space time as long as the variation between \(a\) and \(b\) occurs linearly in spacetime, so vanishes when we take the second derivative.
Hence,
\begin{equation}
    \sum_{\lambda = 1}^{4} j_\mu^{(\Pe)} j_\nu^{(\Ptau)} \varepsilon_\mu^{(\lambda)}(\varepsilon_\nu^{(\lambda)})^* = \sum_{\lambda = 1}^4 j_\mu^{(\Pe)} j_\nu^{(\Ptau)} (\varepsilon_\mu^{(\lambda)} + aq^\mu)((\varepsilon_\nu^{(\lambda)})^* + bq^\nu)
\end{equation}
for all \(a\) and \(b\).
This means we have
\begin{equation}
    j_\mu^{(\Pe)} j_\nu^{(\Ptau)} q^\mu q^\nu = 0.
\end{equation}
We can then conclude that the \(q_\mu q_\nu/q^2\) term does not contribute to the amplitude.
Hence, we have
\begin{equation}\label{eqn:spin sum}
    \sum_{\lambda = 1}^{4} \varepsilon^{(\lambda)}_\mu (\varepsilon_\nu^{(\lambda)})^* = -\minkowskiMetric^{\mu\nu}.
\end{equation}

The Feynman rule for a photon propagator in a first order diagram is then
\begin{equation}
    -i\frac{\minkowskiMetric_{\mu\nu}}{q^2}.
\end{equation}
For higher order diagrams we cannot guarantee the gauge invariance of a single diagram's contribution to the amplitude and instead the photon propagator is
\begin{equation}
    -\frac{i}{q^2}\left[ \minkowskiMetric_{\mu\nu} + (1 - \xi) \frac{q^\mu q^\nu}{q^2} \right]
\end{equation}
where \(\xi\) is a gauge dependent parameter.
Most calculations are performed in the Feynman gauge in which \(\xi = 1\).