\chapter{Hamiltonian Mode Expansion}\label{app:hamiltonian mode expansion}
The Hamiltonian for the free Klein--Gordon field is
\begin{equation}
    H = \frac{1}{2} \int \dl{^3x} \left\{ (\pi(x))^2 + (\grad \varphi(x))^2 + m^2(\varphi(x))^2 \right\} \equiv H_1 + H_2 + H_3.
\end{equation}
In \cref{sec:the hamiltonian} we claimed that the time independent contribution from each term is
\begin{align}
    H_1 &\to \frac{1}{8} \int \frac{\dl{^3p}}{(2\pi)^3} \left\{ a(\vv{p}) a^\hermit(\vv{p}) + a^\hermit(\vv{p})a(\vv{p}) \right\},\\
    H_2 &\to \frac{1}{8} \int \frac{\dl{^3p}}{(2\pi)^3} \frac{\vv{p}^2}{\omega(\vv{p})^2} \left\{ a(\vv{p})a^\hermit(\vv{p}) + a^\hermit(\vv{p})a(\vv{p}) \right\},\\
    H_3 &\to \frac{1}{8} \int \frac{\dl{^3p}}{(2\pi)^3} \frac{m^2}{\omega(\vv{p})^2} \left\{ a(\vv{p})a^\hermit(\vv{p}) + a^\hermit(\vv{p})a(\vv{p}) \right\}.
\end{align}
We also claimed that the time dependent parts cancel in \(H_1 + H_2 + H_3\).
In this section we complete the full calculation showing this to be true. 

Start with the mode expansion of \(\varphi(x)\):
\begin{equation}
    \varphi(x) = \int \frac{\dl{^3p}}{(2\pi)^3} \frac{1}{2\omega(\vv{p})} [a(\vv{p})\e^{-ip\cdot x} + a^\hermit(\vv{p}) \e^{ip\cdot x}].
\end{equation}
Taking the time derivative we get
\begin{equation}
    \pi(x) = \dot{\varphi}(x) = -\frac{i}{2} \int \frac{\dl{^3p}}{(2\pi)^3} [a(\vv{p})\e^{-ip\cdot x} - a^\hermit(\vv{p})\e^{ip\cdot x}].
\end{equation}
We can then compute \(\pi^2\):
\begin{multline}
    \pi(x)^2 = -\frac{1}{4} \int \frac{\dl{^3p}}{(2\pi)^3} \int \frac{\dl{^3p'}}{(2\pi)^3} [a(\vv{p})\e^{-ip\cdot x} - a^\hermit(\vv{p})\e^{ip\cdot x}]\\
    [a(\vv{p}')\e^{-ip'\cdot x} - a^\hermit(\vv{p}')\e^{ip'\cdot x}]
\end{multline}
Expand the integrand to get
\begin{multline}
    a(\vv{p})a(\vv{p}')\e^{-i(p + p') \cdot x} - a(\vv{p})a^\hermit(\vv{p}') \e^{-i(p - p')\cdot x}\\
    - a^\hermit(\vv{p}) a(\vv{p}') \e^{i(p - p')\cdot x} + a^\hermit(\vv{p})a^\hermit(\vv{p}')\e^{i(p + p')\cdot x}
\end{multline}
Now consider the identities
\begin{align}
    \int \dl{^3x} \, \e^{\pm i (p + p')\cdot x} &= (2\pi)^3\delta(\vv{p} + \vv{p}') \e^{\pm 2i\omega(\vv{p})t}\\
    \int \dl{^3x} \, \e^{\pm i(p - p')\cdot x} &= (2\pi)^3\delta(\vv{p} - \vv{p}').
\end{align}
With these we can perform the \(x\) integral to get
\begin{multline}
    H_1 = -\frac{1}{8} \int \frac{\dl{^3p}}{(2\pi)^3} \int \frac{\dl{^3p'}}{(2\pi)^3} [a(\vv{p})a(\vv{p}')(2\pi)^3\delta(\vv{p} + \vv{p}')\e^{-2i\omega(\vv{p})t}\\
    \qquad- a(\vv{p})a^\hermit(\vv{p}') (2\pi)^3 \delta(\vv{p} - \vv{p}') - a^\hermit(\vv{p}) a(\vv{p}') (2\pi)^3 \delta(\vv{p} - \vv{p}')\\
    + a^\hermit(\vv{p})a^\hermit(\vv{p}') (2\pi)^3\delta(\vv{p} + \vv{p}')\e^{2i\omega(\vv{p})t}].
\end{multline}
Now performing the \(p'\) integral we get
\begin{multline}
    H_1 = -\frac{1}{8} \int \frac{\dl{^3p}}{(2\pi)^3} [a(\vv{p})a(-\vv{p})\e^{-2i\omega(\vv{p})} - a(\vv{p})a^\hermit(\vv{p})\\
    + a^\hermit(\vv{p}) a(\vv{p}) + a^\hermit(\vv{p})a^\hermit(-\vv{p})\e^{2i\omega(\vv{p})t}].
\end{multline}
The time independent part of this is
\begin{equation}
    H_1 \to \frac{1}{8} \int \frac{\dl{^3p}}{(2\pi)^3} [a(\vv{p})a^\hermit(\vv{p}) + a^\hermit(\vv{p})a(\vv{p})].
\end{equation}
The time dependent part is
\begin{equation}
    H_1^t = -\frac{1}{8} \int \frac{\dl{^3p}}{(2\pi)^3} [a(\vv{p})a(-\vv{p}) \e^{-2i\omega(\vv{p})} + a^\hermit(\vv{p})a^\hermit(-\vv{p}) \e^{2i\omega(\vv{p})t}].
\end{equation}
This should (hopefully) cancel out with the other two terms.

Now consider the second term.
We start by computing \(\grad \varphi\) in terms of the mode operators, which gives
\begin{equation}
    \grad \varphi(x) = \frac{i}{2}\int \frac{\dl{^3p}}{(2\pi)^3} \frac{\vv{p}}{\omega(\vv{p})} [a(\vv{p})\e^{-ip\cdot x} - a^\hermit(\vv{p})\e^{ip\cdot x}].
\end{equation}
We can then compute \((\grad \varphi)^2\):
\begin{multline}
    (\grad\varphi(x))^2 = -\frac{1}{4} \int \frac{\dl{^3p}}{(2\pi)^3} \frac{\vv{p}\cdot \vv{p}'}{\omega(\vv{p})\omega(\vv{p}')} [a(\vv{p})\e^{-ip\cdot x} - a^\hermit(\vv{p})\e^{ip\cdot x}]\\
    [a(\vv{p}')\e^{-ip'\cdot x} - a^\hermit(\vv{p}')\e^{ip'\cdot x}].
\end{multline}
Expanding the integrand we get
\begin{multline}
    a(\vv{p})a(\vv{p}')\e^{-i(p + p')\cdot x} - a(\vv{p})a^\hermit(\vv{p}') \e^{-i(p - p')\cdot x}\\
    - a^\hermit(\vv{p}) a(\vv{p}')\e^{i(p - p')\cdot x} + a^\hermit(\vv{p})a^\hermit(\vv{p}')\e^{i(p + p')\cdot x}.
\end{multline}
We can then perform the integral over \(x\) and we'll get Dirac deltas:
\begin{multline}
    H_2 = -\frac{1}{8} \int \frac{\dl{^3p}}{(2\pi)^3} \int \frac{\dl{^3p'}}{(2\pi)^3} \frac{\vv{p}\cdot \vv{p}'}{\omega(\vv{p})\omega(\vv{p}')} [a(\vv{p})a(\vv{p}')(2\pi)^3\delta(\vv{p} + \vv{p}')\e^{-2i\omega(\vv{p})t}\\
    \qquad- a(\vv{p})a^\hermit(\vv{p}') (2\pi)^3\delta(\vv{p} - \vv{p}') - a^\hermit(\vv{p}) a(\vv{p}')(2\pi)^3\delta(\vv{p} - \vv{p}')\\
    + a^\hermit(\vv{p})a^\hermit(\vv{p}')(2\pi)^3\delta(\vv{p} + \vv{p}')\e^{2i\omega(\vv{p})t}]
\end{multline}
Performing the \(p'\) integral we get
\begin{multline}
    H_2 = -\frac{1}{8} \int \frac{\dl{^3p}}{(2\pi)^3} \frac{\abs{\vv{p}}^2}{\omega(\vv{p})^2} [-a(\vv{p})a(-\vv{p})\e^{-2i\omega(\vv{p})t} - a(\vv{p})a^\hermit(\vv{p})\\
    - a^\hermit(\vv{p}) a(\vv{p}) - a^\hermit(\vv{p})a^\hermit(-\vv{p})\e^{2i\omega(\vv{p})t}].
\end{multline}
Note the extra negative sign from setting \(\vv{p} = -\vv{p}\) in the factor out the front for the time dependent parts.
The time independent part of this is
\begin{equation}
    H_2 \to \int \frac{\dl{^3p}}{(2\pi)^3} \frac{\abs{\vv{p}}^2}{\omega(\vv{p})^2} [a(\vv{p})a^\hermit(\vv{p}) + a^\hermit(\vv{p})a(\vv{p})].
\end{equation}
The time dependent part is
\begin{equation}
    H_2^t = \frac{1}{8} \int \frac{\dl{^3p}}{(2\pi)^3} \frac{\abs{\vv{p}}^2}{\omega(\vv{p})^2} [a(\vv{p})a(-\vv{p})\e^{-2i\omega(\vv{p})t} + a^\hermit(\vv{p})a^\hermit(-\vv{p})\e^{2i\omega(\vv{p})t}].
\end{equation}

Finally, consider the \(\varphi^2\) term:
\begin{multline}
    \varphi(x)^2 = \frac{1}{4} \int \frac{\dl{^3p}}{(2\pi)^3} \int \frac{\dl{^3p'}}{(2\pi)^3} \frac{1}{\omega(\vv{p})\omega(\vv{p}')} [a(\vv{p})\e^{-ip\cdot x} + a^\hermit(\vv{p})\e^{ip\cdot x}]\\
    [a(\vv{p}')\e^{-ip'\cdot x} + a^\hermit(\vv{p}')\e^{ip'\cdot x}].
\end{multline}
Expand the integrand to get
\begin{multline}
    a(\vv{p})a^\hermit(\vv{p}')\e^{-i(p + p')\cdot x} + a(\vv{p})a^\hermit(\vv{p}')\e^{-i(p - p')\cdot x}\\
    + a^\hermit(\vv{p})a(\vv{p}')\e^{i(p - p')\cdot x} + a^\hermit(\vv{p})a^\hermit(\vv{p}')\e^{i(p + p')\cdot x}.
\end{multline}
We can then perform the \(x\) integral to get Dirac deltas:
\begin{multline}
    H_3 = \frac{1}{8} \int \frac{\dl{^3p}}{(2\pi)^3} \int \frac{\dl{^3p'}}{(2\pi)^3} \frac{m^2}{\omega(\vv{p})\omega(\vv{p}')} [a(\vv{p})a(\vv{p}')(2\pi)^3\delta(\vv{p} + \vv{p}')\e^{-2i\omega(\vv{p})t}\\
    \qquad+ a(\vv{p})a^\hermit(\vv{p}')(2\pi)^3\delta(\vv{p} - \vv{p}') + a^\hermit(\vv{p})a(\vv{p}')(2\pi)^3\delta(\vv{p} - \vv{p}')\\
    + a^\hermit(\vv{p})a^\hermit(\vv{p}')(2\pi)^3\delta(\vv{p} + \vv{p}')\e^{2i\omega(\vv{p})t}]
\end{multline}
Performing the \(p'\) integral we get
\begin{multline}
    H_3 = \frac{1}{8} \int \frac{\dl{^3p}}{(2\pi)^3} \frac{m^2}{\omega(\vv{p})^2} [a(\vv{p})a(-\vv{p})\e^{-2i\omega(\vv{p})} + a(\vv{p})a^\hermit(\vv{p})\\
    + a^\hermit(\vv{p})a(\vv{p}) + a^\hermit(\vv{p})a^\hermit(-\vv{p})\e^{2i\omega(\vv{p})t}]
\end{multline}
The time independent part of this is
\begin{equation}
    H_3 \to \frac{1}{8}\int\frac{\dl{^3p}}{(2\pi)^3} \frac{m^2}{\omega(\vv{p})^2} [a(\vv{p})a^\hermit(\vv{p}) + a^\hermit(\vv{p})a(\vv{p})].
\end{equation}
The time dependent part is
\begin{equation}
    H_3^t = \frac{1}{8} \int \frac{\dl{^3p}}{(2\pi)^3} \frac{m^2}{\omega(\vv{p})^2} [a(\vv{p})a(-\vv{p})\e^{-2i\omega(\vv{p})t} + a^\hermit(\vv{p})a^\hermit(-\vv{p})\e^{2i\omega(\vv{p})t}].
\end{equation}

Now we consider the time dependent parts.
First notice that each has a factor of
\begin{equation}
    \frac{1}{8} \int \frac{\dl{^3p}}{(2\pi)^3} [a(\vv{p})a(-\vv{p})\e^{-2i\omega(\vv{p})t} + a^\hermit(\vv{p})a^\hermit(-\vv{p})\e^{2i\omega(\vv{p})}],
\end{equation}
so we focus on the terms not present here:
\begin{equation}
    -1 + \frac{\abs{\vv{p}}^2}{\omega(\vv{p})^2} + \frac{m^2}{\omega(\vv{p})^2} = \frac{\abs{\vv{p}}^2 + m^2}{\omega(\vv{p})^2} - 1 = 1 - 1 = 0
\end{equation}
since by definition \(\omega(\vv{p})^2 = \abs{\vv{p}}^2 + m^2\).