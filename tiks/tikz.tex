\documentclass[12pt]{article}
\usepackage[lmargin=0.5cm,rmargin=1cm,top=0.5cm,bottom=0.5cm]{geometry}
\usepackage[utf8]{inputenc}
\usepackage[T1]{fontenc}
\usepackage[spanish]{babel}
%\parident=0cm
%%
\usepackage{tikz}
%%

\begin{document}

\section{Tikz}
\subsection{Figuras junto a texto.}
Iniciamos con una sencilla figura \tikz{\draw(0,0) rectangle(1,0.5)}.\\[0.5cm]

Ejemplo en tikz colocando
\tikz[scale=0.25]{
\draw[help lines] (0,-1) grid(4,2);
\draw[fill=blue] (0,0) rectangle (2,1);
\fill[ball color=red](3,0) circle(1);

}
Figuras junto a texto.
\tikz{
\draw (0,0) -- (2,0);


}



\end{document}